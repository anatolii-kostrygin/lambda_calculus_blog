\documentclass[12pt, a4paper]{article}
\usepackage{amsmath,amsthm,amssymb}
\usepackage[utf8x]{inputenc}
\newcommand{\lft}{\leftarrow}
\newcommand{\rgt}{\rightarrow}
\begin{document}
\subsection*{Histoire de $\lambda$-calcul}

Qu'est-ce que c'est un $\lambda$-calcul ?
Les codeurs sont familiers avec une notion de $\lambda$-fonction - une fonction anonyme qui est utilisée dans les morceaux du code qui ne méritent pas d'avoir un méthode nommé : clé de trie, les petits transformation dans les requêtes à la sql etc.
Cépendant, la notion de $\lambda$-fonction a été pris d'une système de calcul aussi puisant que la machine de Turing (est inventé dans les mêmes années 30s).
Dans cet article on va discuter l'histoire de son invention pour mieux comprendre le concept

\subsubsection*{Plan}
\begin{enumerate}
	\item Crise des fondaments
	\item Axiomes de Peano
	\item 1ere version de Lambda calculs
	\item Theoreme de Goedel
	\item Machine de Turing et calculabilité
	\item Thèse de Church-Turing
	\item Applications
\end{enumerate}


\subsubsection*{Crise des fondaments}
À la fin de XX siècle les mathématiciens ont construit une théorie des ensemble : tout le monde est familier avec ça version naïve depuis l'ecole. Ce désir de formaliser ses fondaments a donne lieu des travaux comme les livres de Nicolas Bourbaki, spécialement connu pour sa définition de zéro. $/spoiler/$. Cépendant, on est vite tombé sur les nombreux paradoxes.  Considerons par example un paradoxe de Russel qui réclame que l'ensemble des ensembles n'appartenant pas à eux-mêmes est impossible. Effectivement, considerons $X := \{t|t\notin t\}$. Soit $X \in X$, soit $X \notin X$, mais l'un est l'autre sont contradictoire avec une définition de $X$. Evidement, l'existence de ce paradoxe relève des doutes sur les autres domains de mathématique : et si on trouve les même paradoxes en analyse, algèbre etc.? Ce période est connu comme une "Crise des fondements".

La solution a été de "revisiter" les bases de mathématique est étudier l'existance eventuelle des autres paradoxes. Ce crise a été reflété sous le numéro 2 dans une liste des 23 fameux problèmes de Hilbert qui ont détérminé le développement du mathématique en XXème siècle. Plus précisement, la deuxième problème a été de déterminer la consistence de l'arithmétique. Même si l'enoncé est simple, la solution nécessite de passer par deux grands étapes :
- formuler les axiomes de l'arithmétique - i.e., trouver les proposition "minimales" telles que on peut en déduire tout ce que l'on connait jusqu'qu présent.
- prouver que en partant de ces axiomes, il n'existe pas d'une proposition X, tel que les axiome implique X ainsi que "non X".
Pour justifier l'importance de ce problème, voici quelques noms des mathématiciens qui ont contribuer dans la solution: Péano, Dedekind, Gödel, Church, Rassel, Kleene, Rosser etc.


\subsubsection*{Axiomes de Peano}
Le résultat ?
1) Aux années 1889, Peano a proposé les axiomes de l'arithmétique les plus connues : ...

\subsubsection*{1ère version de Lambda calcul}
2) Vers les années 1932 Church a proposé une autre construction qui a été appelé lambda-calcul. Malheureusement, son étudiant, Kleene a prouvé que cette construction n'a pas été consistente.

$\lambda$-calcul a formalisé une application d'une fonction. L'écriture envisage la compréhension d'une fonction comme une "règle". Et l'écriture classique $f(x)$ pointe plutôt sur le résultat de ce règle.

\subsubsection*{Théorème de Gödel}
3) Le résultat le plus connu, a été prouvé par Kurt Gödel : il a prouvé que la consistence d'une système d'axiomes ne peut pas être prouvée en n'utilisent que ces axiomes. Donc pour prouver la consistence d'arithmétique il faut ajouter les axiomes supplémentaires (qui a été vite fait, en 1936). Le seul problème est que maintenant il faut prouver une autre système...

Pour résumer le sujet de l'arithmétique, disons que lambda-calcul a été l'un des modèles qui pourrait formaliser les axiomes de l'arithmétique. Son version actuel a été prouvée consiétente et publiée en 1936. Cette construction devait rester un sujet purement théorique qui a intéressé les rares genies de mathématique qui a étudié ses fondaments (on rappel que beaucoup des personnages de cet article se sont finis mal...). 

\subsubsection*{Machine de Turing et calculabilité}
Cépéndant, comme il est souvent en science, il faudrait étudier le même domaine de point de vue un peu différent. Cela a été fait sur l'autre continent par un jeun étudiant Alan Turing. Il a cherché une solution pour une problème de la décision posé en 1928 par Hilbert et Ackermann : "trouver un algorithme qui détermine dans un temps fini, s'il un énoncé est vrai ou faux". La formalisation d'un terme algorithme a conduit au concept de machine de Turing connu par tout le monde. Entre outre, le théorème de Gödel a été reformuler en thermes d'une machine de Turing.
Le résultat a été aussi négative, connu comme une théorème de Turing-Church: "il existe les énoncés pour lesquels on ne peux pas déterminer" (vérifier l'enoncé et le nom d'un théorème).

\subsubsection*{Thèse de Church-Turing}
Le résultat positif.
S'il existe les fonctions, qu'il peuvent pas être décidées, on se pose la question, qu'est-ce que ce sont les fonction simple, i.e. les fonctions que l'on peut effectivement calculer.
Intuitivement, c'est dont la valeur peut être calculée avec un crayon si on a suffisament de papier et du temps.
Mais vous comprenez déjà que les mathématiciens n'acceptent pas les solutions intuitives...
Le problème de décision est lié avec une problème de calculabilité. Qu'est-ce que signifie qu'une fonction peut être calculée ? Souvent on se refère sur "des méthodes d'un crayon et de papier". Indépendement, chaque des deux a proposé que toute fonction calculable en thèrmes de crayon et papier peut être calculé par son méthode (lambda-calcul ou la machine de Turing). Les deux propositions - ne sont pas les théorèmes, peuvent pas être prouvées car on ne peut pas formaliser autrement calculabilité. (On doit remarquer ici qu'il y avait le troisième mechanisme de détérminer la calculabilité - les fonction récursive primitive). Relativement vite il a été prouvé que tout les 3 méchanismes sont équivalent. Donc, n'importe lequel peut être utilisé comme une définition de function effectivement calculable.

\subsubsection*{Impacts de $\lambda$-calculs}
\begin{enumerate}
	\item Formalisation d'une notion de calculabilité.
	\item Preuves de calculabilité.
	\item Preuves formelles.
	\item Programmation fonctionnelle.
\end{enumerate}

\end{document}
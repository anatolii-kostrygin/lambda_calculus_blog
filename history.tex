\documentclass[12pt, a4paper]{article}
\usepackage{amsmath,amsthm,amssymb}
\usepackage[utf8x]{inputenc}
\newcommand{\lft}{\leftarrow}
\newcommand{\rgt}{\rightarrow}
%\newtheorem{problem}{Problème}
\newtheorem*{problem}{Problème}
\begin{document}
\subsection*{Histoire de $\lambda$-calcul}

Qu'est-ce que c'est un $\lambda$-calcul ?
Les codeurs sont familiers avec une notion de $\lambda$-fonction - une fonction anonyme qui est utilisée dans les morceaux du code qui ne méritent pas d'avoir un méthode nommé : clé de trie, les petits transformation dans les requêtes à la sql etc.
Cépendant, la notion de $\lambda$-fonction a été pris d'une système de calcul aussi puisant que la machine de Turing (est inventé dans les mêmes années 30s).
Dans cet article on va discuter l'histoire de son invention pour mieux comprendre le concept.

\subsubsection*{Plan}
\begin{enumerate}
	\item Crise des fondaments
	\item Axiomes de Peano
	\item 1ere version de Lambda calculs
	\item Theoreme de Goedel
	\item Machine de Turing et calculabilité
	\item Thèse de Church-Turing
	\item Impact et applications
\end{enumerate}


\subsubsection*{Crise des fondaments}
Disons qu'un ensemble est \textbf{simple} s'il n'appartient pas à lui-même. Par exemple, l'ensemble des tout les gens est simple, car cet ensemble n'est pas une personne. Ainsi, l'ensemble des tout les ensemble n'est pas simple par son définition. \emph{L'ensemble de Russel} est un ensemble qui contient tout les ensembles simples et rien d'autre.
Est-ce qu'un ensemble de Russel est simple ? Si c'est le cas, par construction il contient lui-même. Donc il n'est pas simple. Mais s'il n'est pas simple il doit contenir lui-même, ce que signifie qu'il est simple. \emph{Contradiction}.

Ce paradoxe a été indépendement trouvé par Russel et Zermelo au début de XX siècle. Ce paradoxe a beaucoup d'autre formulations plutôt didactiques : \emph{paradoxe du menteur}, \emph{paradoxe du barbier}, etc. 
% link => Martin Gardner. Aha! Gotcha. Paradoxes to puzzle and delight. 
Cépendant dans une version de Russel ce paradoxe n'utilise que les constructions formelles de mathématique. Cela signifie que telles constructions sont contradictoires elles-mêmes : si nous avons prouvé qu'une formule propositionnelle est à la fois vrai et fausse, le même peut avoir lieu pour n'importe quel théorème.
Si on ajoute qu'au début de XXème siècle paradoxe de Russel n'a pas été le seul paradoxe connu, on voit bien qu'est-ce que c'était le \emph{crise de fondaments} en mathématique.

Ce crise a été refleté sous le numéro 2 dans une liste des 23 fameux problèmes de Hilbert détérminants le développement du mathématique en XXème siècle. 
\begin{problem}[2ème problème de Hilbert]
	Déterminer la consistence de l'arithmétique.
\end{problem}
Même si l'enoncé est simple, on peut dire que pour résoudre ce problème, il faut passer par les étapes suivants.
\begin{itemize}
	\item Formuler les axiomes de l'arithmétique - i.e., trouver les proposition "minimales" telles que on peut en déduire tout ce que l'on connait jusqu'qu présent.
	\item Prouver que en partant de ces axiomes, il n'existe pas d'une proposition X, tel que les axiome implique X ainsi que "non X".
\end{itemize}


%À la fin de XIX siècle les mathématiciens ont construit une théorie des ensemble : tout le monde est familier avec ça version naïve depuis l'ecole. Ce désir de formaliser ses fondaments a donne lieu des travaux comme les livres de Nicolas Bourbaki, spécialement connu pour sa définition de zéro. $/spoiler/$. Cépendant, on est vite tombé sur les nombreux paradoxes.  Considerons par example un paradoxe de Russel qui réclame que l'ensemble des ensembles n'appartenant pas à eux-mêmes est impossible. Effectivement, considerons $X := \{t|t\notin t\}$. Soit $X \in X$, soit $X \notin X$, mais l'un est l'autre sont contradictoire avec une définition de $X$. Evidement, l'existence de ce paradoxe relève des doutes sur les autres domains de mathématique : et si on trouve les même paradoxes en analyse, algèbre etc.? Ce période est connu comme une "Crise des fondements".
%
%La solution a été de "revisiter" les bases de mathématique et étudier l'existance eventuelle des autres paradoxes. Ce crise a été reflété sous le numéro 2 dans une liste des 23 fameux problèmes de Hilbert qui ont détérminé le développement du mathématique en XXème siècle. Plus précisement, la deuxième problème a été de déterminer la consistence de l'arithmétique. Même si l'enoncé est simple, la solution nécessite de passer par deux grands étapes :
%- formuler les axiomes de l'arithmétique - i.e., trouver les proposition "minimales" telles que on peut en déduire tout ce que l'on connait jusqu'qu présent.
%- prouver que en partant de ces axiomes, il n'existe pas d'une proposition X, tel que les axiome implique X ainsi que "non X".
%Pour justifier l'importance de ce problème, voici quelques noms des mathématiciens qui ont contribuer dans la solution: Péano, Dedekind, Gödel, Church, Rassel, Kleene, Rosser etc.


\subsubsection*{Axiomes de Peano}
L'arithmétique est un domain de mathématique qui étude les nombres et rélations entre eux.
Elle est appliquée partout de premiers années de l'école jusqu'à les conceptes modernes d'astrophysique.
Cépendant, pour construire les bases de l'arithmétique, il est presque suffisant de bien déterminer les nombres naturels ainsi que les action qu'on peut faire avec.
(Les nombres entiers est une extension pour que opération $x - y$ renvoit toujours un nombre valide, les nombres rationnels aparaissent si on étudie la division. Finalement, les nombres algébrique sont résponsable pour résoudre les équatinos polynomials et le reste - pour "fermer des trous").
Classiquement, les nombres naturels peuvent être définis de même façon qu'on fait quand les petits enfants apprennent à compter: ce résultat est connus dépuis la find de XIXème siècle comme les axiomes de Péano:
\begin{enumerate}
	\item 1 est naturel;
	\item le nombre suivant d'un nombre naturel est naturel;
	\item rien ne suivi de 1;
	\item si $a$ suive $b$ et $a$ suive $c$, alorc $b=c$;
	\item axiome de recurrence (i.e., si un prédicat $A(x)$ est vrai pour $x=1$ ainsi que $A(n)$ implique $A(n+1)$, alors $A(x)$ est vrai pour tout $n$ naturel).
\end{enumerate}
Heureusement, la preuve d'une consistence des axiomes de Péano est un problème beaucoup plus sophistiqué que l'invention de ses axiomes, et l'histoire n'est donc que commencée.

\subsubsection*{1ère version de Lambda calcul}
En 1932 Church a proposé une autre construction qui est connue comme \textbf{$\lambda$-calcul non-typé}.
Malheureusement, son étudiant, Kleene a prouvé que cette construction n'a pas été consistente.

$\lambda$-calcul a formalisé une application d'une fonction. L'écriture envisage la compréhension d'une fonction comme une "règle". Et l'écriture classique $f(x)$ pointe plutôt sur le résultat de ce règle.

Rappelon brièvement, qu'est ce que c'est. (Sinon, wiki et les autres articles ou "Eggs and crocodiles")
Le brique principal est une fonction. 
Au lieu de $f(x)$ on écrit $\lambda x.f$.
Si on parle de la valeur de $f(x)$ quand $x=a$, on écrit $\lambda x.f a$.
Naturelement, ion peut définir une composition...
Pour transformer des propositions on a une règle de $\beta$-reduction.

Malgré sa simplicité et abstraité, cette construction permet néanmoins rédefinir tout les opérations arithmétiques, la logique Booléen etc.
Est-ce que $\lambda$-calcul non-typé est un bon candidat pour le rôle de fondament de mathématique ?
La réponse est \textbf{non}: à cause de Paradoxe de Kleene-Rosser proposé en 1935 par J. B. Rosser et Stephen Kleene qui a été un étudiant de Church.
Bien que la propre énoncé de ce paradoxe est trop compliqué pour cet article, nous pouvons décrire la raison d'un problème.
Commençons par une phrase "si cette phrase est vrai, alors $X$", où $X$ est un énoncé qui est évidement faux, e.g., "Allemand et Chine ont une frontière commune".
Après, par une analyse logique (\textbf{todo}), on peut déduire que n'importe quel énoncé $X$ est vrai.
C'est une version non-formel de paradoxe basée sur l'auto-référence.
Il peut être formulé en termes de $\lambda$-calculs.

{\footnotesize
	Considerons une fonction $r$ définie comme $r=\lambda x.((x x) \to y)$.
	$(r r)$ $\beta$-se réduit en $(r r) \to y$.
	Si $(r r)$ est faux, alors $(r r) \to y$ est vrai par le principe d'explosion, mais cela est contradictoire avec la $\beta$-réduction.
	Donc $(r r)$ est vrai.
	On en déduit que $y$ est aussi vrai.
	Comme $y$ peut être arbitraire, on a prouvé que n'importe quel proposition est vrai.
	Contradiction.
}

\subsubsection*{Théorème de Gödel}
Les deux paradoxes discutés ci-dessus, sont basés sur le même idée de l'autoréference : une proposition ou n'importe quel objet qui réference lui-même (e.g., ensemble des tout les ensembles).
Faut-il intérdir l'autoréference dans les constructions mathématiques ?
L'idée n'est pas séduisant si on comprends vite que avec les paradoxes, nous avons jeté dans la poubelle tout les construcitons récursives.

Néanmoins, les réflections de mathématiciens de même époque nous ammené à un résultat le plus connu prouvé par Kurt Gödel en 1930. Une des intérpretations pretends que la consistence d'une système d'axiomes ne peut pas être prouvée en n'utilisent que ces axiomes (voici l'autoréference !). En particulier, pour prouver la consistence d'arithmétique il faut ajouter les axiomes supplémentaires (qui a été vite fait, en 1936). Le seul problème est que maintenant il faut prouver une autre système...
{\footnotesize
	Les idées de l'autoréference sont vraiment fondaméntales.
	Si quelqu'un veut creuser ce sujet plus profondement, nous pouvons conseiller un livre "Gödel, Escher, Bach : Les Brins d'une Guirlande Éternelle" de Douglas Hofstadter.
}

Pour résumer le sujet de l'arithmétique, disons que lambda-calcul a été l'un des modèles qui pourrait formaliser les axiomes de l'arithmétique. Son version actuel a été prouvée consiétente et publiée en 1936. Cette construction devait rester un sujet purement théorique qui a intéressé les rares genies de mathématique qui a étudié ses fondaments (on rappel que beaucoup des personnages de cet article se sont finis mal...). 

\subsubsection*{Machine de Turing et calculabilité}
Cépéndant, comme il est souvent en science, il faudrait étudier le même domaine de point de vue un peu différent. Cela a été fait sur l'autre continent par un jeun étudiant Alan Turing. Il a cherché une solution pour une problème de la décision posé en 1928 par Hilbert et Ackermann : "trouver un algorithme qui détermine dans un temps fini, s'il un énoncé est vrai ou faux". La formalisation d'un terme algorithme a conduit au concept de machine de Turing connu par tout le monde. Entre outre, le théorème de Gödel a été reformuler en thermes d'une machine de Turing.
Le résultat a été aussi négative, connu comme une théorème de Turing-Church: "il existe les énoncés pour lesquels on ne peux pas déterminer" (vérifier l'enoncé et le nom d'un théorème).

\subsubsection*{Thèse de Church-Turing}
Le résultat positif.
S'il existe les fonctions, qu'il peuvent pas être décidées, on se pose la question, qu'est-ce que ce sont les fonction simple, i.e. les fonctions que l'on peut effectivement calculer.
Intuitivement, c'est dont la valeur peut être calculée avec un crayon si on a suffisament de papier et du temps.
Mais vous comprenez déjà que les mathématiciens n'acceptent pas les solutions intuitives...
Le problème de décision est lié avec une problème de calculabilité. Qu'est-ce que signifie qu'une fonction peut être calculée ? Souvent on se refère sur "des méthodes d'un crayon et de papier". Indépendement, chaque des deux a proposé que toute fonction calculable en thèrmes de crayon et papier peut être calculé par son méthode (lambda-calcul ou la machine de Turing). Les deux propositions - ne sont pas les théorèmes, peuvent pas être prouvées car on ne peut pas formaliser autrement calculabilité. (On doit remarquer ici qu'il y avait le troisième mechanisme de détérminer la calculabilité - les fonction récursive primitive). Relativement vite il a été prouvé que tout les 3 méchanismes sont équivalent. Donc, n'importe lequel peut être utilisé comme une définition de function effectivement calculable.

\subsubsection*{Impacts de $\lambda$-calculs}
\begin{enumerate}
	\item Formalisation d'une notion de calculabilité.
	\item Preuves de calculabilité.
	\item Preuves formelles.
	\item Programmation fonctionnelle.
\end{enumerate}

\end{document}